These tables provide random encounters for the regions of Theaceae. The encounters can be adjusted as desired to account for party level or narrative needs.

\subsection{Costal Region}

For random encounters on the east side of the mountains, roll 1d8 and 1d12.

\rowcolors{1}{gray!25}{white}
\begin{tabular}{|c|c|}

\hline

\textbf{d12 + d8} & \textbf{Encounter}\\

\hline

2 & Centaur\\

3 & Secret Tea Farm\\

4 & Gnoll Camp\\

5 & Needle Blights\\

6 & Poisonous Snakes\\

7 & False Trail\\

8 & Elvish Ruins\\

9 & Small Outpost\\

10 & Awakened Grove\\

11 & Hunting Stand\\

12 & Trinket\\

13 & Gnoll Pack\\

14 & Trade Convoy\\

15 & Smugglers\\

16 & Dangerous Storm\\

17 & Abandoned Farm\\

18 & Water Elemental\\

19 & Animal Mask Group\\

20 & Druidic Ruins\\

\hline

\end{tabular}

\subsection{Mountains}

For random encounters in the Mountains, roll 1d8 and 1d12.

\rowcolors{1}{gray!25}{white}
\begin{tabular}{|c|c|}

\hline

\textbf{d12 + d8} & \textbf{Encounter}\\

\hline

2 & Pseudodragon\\

3 & Frozen Encampment\\

4 & Herd Of Yaks\\

5 & Elf Traders\\

6 & Herd Of Goats\\

7 & Awakened Grove\\

8 & Trade Convoy\\

9 & Elvish Ruins\\

10 & Trade Convoy\\

11 & Rockslide\\

12 & Trinket\\

13 & Smugglers\\

14 & Dangerous Storm\\

15 & Wolf Pack\\

16 & Dwarf Miners\\

17 & Air Elemental\\

18 & Hidden Mine\\

19 & Roc\\

20 & Dragon Sighting\\

\hline

\end{tabular}

\subsection{Dry Side}

For random encounters on the dry side of the mountains, roll 1d8 and 1d12.

\rowcolors{1}{gray!25}{white}
\begin{tabular}{|c|c|}

\hline

\textbf{d12 + d8} & \textbf{Encounter}\\

\hline

2 & Dryad\\

3 & False Trail\\

4 & Herd Of Elk\\

5 & Will-O'-The-Wisps\\

6 & Herd Of Goats\\

7 & Small Farm\\

8 & Farmer\\

9 & Elvish Ruins\\

10 & Awakened Grove\\

11 & Trinket\\

12 & Hunting Stand\\

13 & Verse Of Sisters\\

14 & Smugglers\\

15 & Tinker\\

16 & Satyrs\\

17 & Dangerous Storm\\

18 & Wolf Pack\\

19 & Fey Hill\\

20 & Dragon Sighting\\

\hline

\end{tabular}

\subsection{Abandoned Farm}

This farm was recently abandoned.
The buildings are usable, but the fields are overgrown with weeds.
If the party investigates, they may discover that the farm was overwhelmed by Gnolls (east side) or evidence of Fey interference (dry side).

\subsection{Air Elemental}

The mountain winds are sometimes more than they seem.
Air elementals are rare, but they are not dangerous unless provoked.

\subsection{Animal Mask Group}

1d6 shifters

A group of people wearing animal furs and masks tries to go unnoticed by the party.
They are not interested in communicating with the party and will attack if the party tries to force contact.
These people demonstrate some animistic skills and tactics when fighting.

\subsection{Awakened Grove}

Occasionally travelers come across a grove of particularly lush and beautiful trees.
Often these trees are perfectly happy to let travels stay nearby if they are polite and quiet.
However, sometimes a grove is particularly mischievous and will pull pranks on the party, such as moving or stealing their equipment.

\subsection{Centaur}

These secretive individuals largely keep to themselves, but sometimes they can be persuaded to divulge information about the forests on the east side.

\subsection{Dangerous Storm}

Large storms are common in Theaceae.
This storm is dangerous enough to prompt the party to take shelter.
Survival checks are needed to find shelter and survive the storm.

\subsection{Dragon Sighting}

A massive shadow briefly passes overhead, and when the party looks, they see a massive red dragon.
If the party attempts to follow the dragon, they see it headed towards the northern end of the White Spine mountains.

\subsection{Druidic Ruins}

The ancient elvish druids had a series of secret outposts scattered throughout the region.
Even now the ruins are very difficult to detect, but observant adventurers can find traces of the entrance.
Many of these ruins have druidic artifacts for controlling the weather and making the land fertile.

\subsection{Dryad}

In particularly deep parts of the forest, dryads can be found taking care of the woods.
So long as the party does not harm any of the trees in front of the dryad, the dryad will leave the party alone.
Adventurers with an affinity for the forest may be able to get information or assistance from the dryad in exchange for small favors.

\subsection{Dwarf Miners}

Dwarves can occasionally be found transporting raw ore to their mountain hold.
Other dwarves regularly bring weapons and armor to trade for gold and supplies.
These groups will happily trade finished goods but will never trade away the raw ore.

\subsection{Elf Traders}

Groups of elves heading to human towns and farms to trade goods are not common, but but they are frequent enough that most people have seen a few elvish trade groups in their town.
As with human trade convoys, they will gladly trade with adventurers on the road.
However, these groups are often suspicious of people on the road and on the lookout for deception or violence.

\subsection{Elvish Ruins}

These elvish ruins are covered moss and vines.
The handful of ruined buildings can offer shelter for the night.
If the buildings are off of the beaten trail, the party may find a magic item or trinket.

\subsection{False Trail}

Between the smugglers, dwarves, fey, and other secretive groups in Theaceae, false trails are annoyingly common off of the main trails.
Many of these false trails end in dead ends, but the fey are particularly fond of making trails that end in traps, dangerous or hilarious depending upon the mood of the fey that made the trail.

\subsection{Farmer}

1d4 commoner(s)

A farmer is bringing a wagon load of produce to the next town for trade or bringing supplies back home.
The farmer will sell any produce but will not sell goods that are needed for the winter.
Any threats of violence will coerce the farmer, but they will curse the party and warn of stories about ill fates for unkind travelers on the road.

\subsection{Fey Hill}

Mysterious hills dot the land.
These hills appear in forests, mountains, and a variety of other locations, but they are always recognizable because they have no foliage except soft green grass year round.
Anyone who sleeps on a fey hill will awake to find themselves under the hill.
These underhill fey homes take on the personality of their owners, with some of the homes being incredibly dangerous and others whimsical.

\subsection{Frozen Encampment}

This encampment has been long abandoned, and only frozen remains of tents and supplies remain.
The supplies are too badly damaged to be usable.
Even if the party investigates, they find no trace of the original inhabitants of the camp.

\subsection{Gnoll Camp}

The Gnolls launch their attacks from a series of moving camps throughout the region.
No effort is made to hide these camps, as the Gnolls are quite happy to attack and eat anything or anyone that stumbles upon their camps.

\subsection{Gnoll Pack}

This pack of Gnolls are loudly making their way through the forest on their way to their next meal.
If these Gnolls find the party, they will attack, hoping to turn the party into their next meal.

\subsection{Herd Of Elk}

Big game animals are rare in Theaceae.
The locals will not hunt these herds, and there are stories of bad fortune falling upon those who do.

\subsection{Herd Of Goats}

Goats are the most common livestock throughout Theaceae.
Shepherds will gladly sell a goat or two when they encounter travelers.

\subsection{Herd Of Yaks}

Herds of yaks are less common than herds of goats.
The animals are prized for their milk and great strength.
These animals are quite expensive.

\subsection{Hidden Mine}

A group of humans have found a vein of ore and established a mine in the mountains.
The dwarves do not take kindly to these mines in their mountains and will destroy any that they find.
As a result, these miners are extremely defensive and will attack anyone that finds them.

\subsection{Hunting Stand}

High in a tree there is a hunting stand using netting and camouflage.
This stand is small, only large enough for two people.

\subsection{Needle Blights}

These small, unpleasant beasts will try attack isolated individuals and carry off their valuables, or even various limbs, before their companions can react.

\subsection{Poisonous Snakes}

Snakes are a common danger in the jungles on the east side of the mountains.
If the adventurers are unfortunate enough to trod upon or disturb the snakes, then they will attack in self defense.

\subsection{Pseudodragon}

Pseudodragons are shy but always curious and excited to see adventurers.
They are eager to play games with the party and acquire new shiny trinkets to hide away in their lair.
If the party befriends a pseudodragon, they may trade magic items they find boring for shiny baubles from the party.

\subsection{Roc}

Rocs have nest high in the White Spine mountains.
When hunting, Rocs tend to sweep down and cart off individual animals or people.
Residents of the towns and farms take cover whenever a Roc is seen in the distance.

\subsection{Rockslide}

Rockslides are common in the mountains.
Acrobatic or athletics checks will be required to safely traverse the trail.

\subsection{Satyrs}

Late in the evening, music and laughter can occasionally be heard coming from the forest.
Adventurers who pursue the music will discover a small group of satyrs partying and merrymaking.
They are happy to have guests join their party, but they will try to entice their guests to join them under their hill, which can be difficult to escape.

\subsection{Secret Tea Farm}

The tea families on the east side tightly control the export of tea from the region.
Given the high prices that these teas fetch, illicit tea growing operations are highly profitable.
These operations are also highly dangerous, with the natural dangers of the east side and the regular patrols from Camellia guards.
Anyone who looks official or works for Camellia will likely be attacked.

\subsection{Small Farm}

These small farms are common on the dry side.
Families that run these farms like to be only close enough to town to get supplies and sell their produce.
They will happily host guests passing by, and often small jobs are available for skilled adventurers.

\subsection{Small Outpost}

Small groups of traders or smugglers sometimes make small outposts off of the beaten path.
While some of these groups just want some isolation, others will aggressively defend the secret of their location.

\subsection{Smugglers}

1d6 bandits + 1 bandit captain

The group of smugglers attempts to disguise themselves as a group of ordinary traders.
With an insight check, the party learns that the group is smuggling a mix of mundane and illicit goods.
Depending upon the smugglers and the side of the mountains, the group may be smuggling elvish artifacts and art or refined teas and medicines.
With an adequate Persuasion or Intimidation check, the group will trade with the adventurers; however, the group will attack at the first sign of threat.

\subsection{Tinker}

A jovial Tinker is very excited to see the party.
They will happily swap stories with the adventurers, and they will engage in strange bartering with the party.
A Tinker rarely has what the party wants but always has what the party needs.

\subsection{Trade Convoy}

A group of predominately human traders and guards come along in a handful of wagons.
The convoy is suspicious of travelers but can be persuaded to engage with the travelers beyond cursory pleasantries if they assure the convoy lead of their good intentions.
Basic supplied and rations can be found at double the prices in the Player's Handbook.

\subsection{Trinket}

A member of the party finds a random trinket on the side of the trail.
Roll on a random trinket table.

\subsection{Verse Of Sisters}

These Sisters are clad in heavy armor and paroling, looking for threats.
They joke and chat back and forth but fall quiet as the adventures come closer.
The Sisters will happily trade information or supplies with the adventurers but will not tolerate any underhanded behavior.

\subsection{Water Elemental}

Some of the lakes and ponds in the region are guarded by water elementals.
A few of these elementals have been given, or assumed, responsibility for ancient magical artifacts.--

\subsection{Will-O'-The-Wisps}

2d6 Will-O'-The-Wisps

The Will-O'-The-Wisps can be seen in the distance, beckoning the party off of the trail.
Party members need to make a wisdom saving throw or become entranced and drawn off the path.
They will attempt to lead the party into traps.
A successful history check may reveal to adventurers that Will-O'-The-Wisps sometimes can be found near graves or treasures.

\subsection{Wolf Pack}

Adventurers will be safe from the packs of wolves in Theaceae so long as they avoid their dens.
There are rumors of some individuals that can take the shape of wolves and live among these packs.
