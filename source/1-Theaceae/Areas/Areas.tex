The plantations on the east side account for the majority of the wealth in the region, but Gnoll raids are frequent throughout the east side, even against large groups.
The city of Camilla in the foothills protects the largest tea plantations and provides the majority of the income for the region.
Most Camillan citizens feel relatively safe inside protection of the iconic white towers known as The Teeth.
The only other city in the east side is the port of Duro, but there are small, isolated groups that try to survive evading the Gnolls and other jungle threats.

The most well known feature of the White Spine mountains is the ruins of the ancient city of Sinensis.
The city sits high in the mountain pass which serves as the main artery between the east side and the dry side.
It is unclear what exactly led to the destruction of Sinensis, but there are rumors of additional ruined Elvish locations throughout the region.

The dry side is generally safer than the rest of the region, but these high altitude planes are less suited to agriculture.
Settlers in this region are glad for the independence and relative isolation the area requires.
Small farms and villages dot the area, with only a handful of large organized groups.
